\section{System Technical Details}\label{sec:flowchart}
\subsection{Block Diagram}
\begin{figure}[tbph]
	\centering
	\includegraphics[width=0.8\linewidth]{"images/block_diagram"}
	\caption{Block diagram of complete system}
	\label{fig:ABCD}
\end{figure}
\newpage
\subsection{Circuit Diagram}
\begin{figure}[tbph]
	\centering
	\includegraphics[width=1\linewidth]{"images/circuit"}
	\caption{Circuit diagram of complete system}
	\label{fig:circuit}
\end{figure}
\newpage
\subsection{System Algorithm/Flowchart}
\begin{figure}[tbph]
	\centering
	\includegraphics[width=0.89\linewidth]{"images/flowchart"}
	\caption{System algorithm/flowchart}
	\label{fig:ABCD}
\end{figure}

\subsection{System Operation}
The inspection system uses external power with an input voltage of 6.6V for the stepper motor and DC motor. Initially, the power to the system is turned on and then the MCU is reset. The system starts with setting the stepper motor to its home position. Then the conveyor belt is started and the inspection system is ready for loading items.

Then, items are loaded onto the conveyor belt in batches, up to 8 items at a time. The items can be placed as close as possible, as the guides provided on the conveyor, will ensure sufficient gap between adjacent items. As the items pass through the sensor stations, the analog sensor reads the reflectance value of the item. The MCU then identifies the type of the item based on the reflectance value and updates the information in a FIFO dynamic linked list. As the item reaches the exit optical sensor, the first item on the linked list is dequeued, and the object type is compared to the current bin position. If required, the conveyor is stopped and the stepper is rotated to adjust the bin to its correct position and then the conveyor belt is restarted and the object is allowed to fall and collected in the appropriate bin.

Due to the limitation of the code in handling the linked list, we had to ensure that the last object of the current batch of 8 items had reached the exit optical sensor before the first item of the next batch of 8 items reached the pre-reflectance optical sensor.

\subsection{Technical Description}
The technical descriptions of the various components and functionalities used in this design are described below:
\begin{itemize}
\item DC Motor\\
A brushed DC motor is used to drive the conveyor belt that which carry the items that are to be sorted. The speed of the motor is controlled using a PWM signal. Dual full bridge driver L298 which can accept standard TTL logic levels is used to drive the DC motor. Two enable inputs are available in the interfacing circuit to enable or disable the DC motor independently of their input signals.

\item Pulse Width Modulation (PWM)\\
Pulse Width Modulation (PWM) helps the MCU to generate a signal with a voltage anywhere between 0V and 3.3V rather than the discrete values of 0V and 3.3V. This can be used to control the input voltage to devices such as speakers, motors, and LEDs. Based on the input voltage the output of the particular device vary. For this project, the PWM signal is given to the DC motor. The speed of the DC motor is dependent on the duty cycle of the PWM signal.

The 8-bit timer of the MCU is used to generate the PWM signal. The timer is set to fast PWM. The counter TCNT0 is used together with OCR0A to set the duty cycle. The PWM signal is received at the PIN 7 of PORT B.

\item Timer Delay\\
A timer function is created to force a delay in the processing. The delay is required for the operation of the stepper motor, for the display of information using LEDs during the testing phase, for the system ramp-down function and for synchronizing the movement of the conveyor belt that makes sure that the previous object is dropped before the belt is stopped for the next item.The 16 bit counter TCNT1 and output compare register OCR1A are used to create a unit delay of 1 milliseconds.

\item Stepper Motor\\
A 6V 0.8A unipolar stepper motor is used to rotate the tray for collecting the falling items in the correct bin. The stepper has a step angle of 1.8 degrees. Though the stepper is unipolar, it is used in a bipolar mode for maximum torque.

\item Analog to Digital Conversion (ADC)\\
Analog reflectance sensor is used to measure the reflectance value of the items to be sorted. 10-bit resolution is used for the ADC. Since the ADC requires a clock frequency between 50 KHz and 200 KHz for maximum resolution, a pre-scale of 16 is selected for the ADC using the register ADCSRA. The ADC result is read from the ADC data registers ADCL and ADCH.ADC interrupt is enabled to trigger an interrupt on completion of one ADC so that another ADC can be started if the item is in the range of the sensor.

\item Pre-Reflectance Optical Sensor\\
It is an active high digital sensor that is used to check if an item is in the range of the reflectance sensor. An ADC is started only when an item is in the range of the analog sensor. This helps to avoid a free running Analog to Digital Conversion which would have taken a lot of CPU time. A filter is used to filter out the noise created by the moving parts used in the project.

\item Reflectance Sensor\\
Reflectance sensor is an analog sensor that measures the reflectance value of the items which are to be sorted. It is the most critical component in identifying the correct type of the object. A 10 bit ADC resolution is used to read the reflectance value of the four objects used in this project. A filter is used in the sensor circuit for filtering out the noise caused by the stepper motor and the DC motor.

\item Exit Optical Sensor\\
It is an active-low digital sensor used to identify if an object has reached the end of the conveyor belt. An interrupt is triggered on the exit optical sensor detection that ensures that the bin is correctly positioned before dropping the item off the conveyor.

\item Hall-Effect Sensor\\
It is an active low digital sensor used to set the bin to its home position. All other bin locations are defined based on their relative positions from the home bin. The tray is rotated 90 degrees clockwise, 90 degrees counterclockwise or 180 degrees as required for adjusting the bin to its correct position before dropping the object from the conveyor belt.

\item System Pause Push-button\\
A push-button is used as the system pause button which disables the DC motor and displays information about the sorted items in each bin and also about the number of items that are partially processed.  The system resumes sorting once the pause button is pressed again.

\item System Ramp-Down Push-Button\\
Another pushbutton is used as the system ramp-down button which when activated, will sort all the items remaining on the conveyor belt and then shuts down the system. The number of objects collected in each bin is displayed on the LCD after the system is shut down. The inspection system needs to be reset before restarting.

\item LCD Display
An LCD display is used to display the number of sorted objects individually by its type and also the number of unsorted objects.
\end{itemize}