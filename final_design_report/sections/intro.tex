\section{Introduction}\label{sec:intro}
\subsection{Problem Statement}
The objective of this design project was to design a high-performance inspection system, implement the proposed design using an ATMEL AVR microcontroller and demonstrate it's functionality. The inspection system was required to sort four types of cylindrically shaped objects which were either black plastic, white plastic, steel, or aluminum material. For the final performance testing, the inspection system needed to sort 48 of such items into their respective bins in less than 60 seconds. Also, there had to be a pause button which paused the inspection system and a ramp-down button which stopped the system after sorting all the items on the conveyor. The details such as the number of each item sorted and unsorted items on the belt were to be displayed using LEDs or optionally using LCD display.

\subsection{Design Purpose and Overview}
The purpose of this design project was to classify and sort 48 cylindrical shaped items which were either black plastic, white plastic, steel, or aluminum in less than 60 seconds. Additionally, a pause button and a system ramp-down button were required to interrupt the system. The AT90USB 1287 microcontroller was used as the heart of the sorting system. A conveyor belt driven by a DC motor was used to carry the items to be sorted through various sensor stations. The sensor stations were used to identify the correct type of the item. Finally, a stepper motor driven bin was used to collect the sorted items while they fell off the conveyor belt. The various components used in the design are described below.

\subsubsection{AT90USB 1287 Microcontroller}
The high-performance, low-power Microchip 8-bit AVR microcontroller combines 128KB ISP flash memory with read-while-write capabilities, 8KB SRAM, 48 general purpose I/O lines, 32 general purpose working registers, real-time counter, four flexible timer/counters with compare modes and PWM, USB 2.0 low-speed and full-speed On-The-Go (OTG) host/device, an 8-channel 10-bit A/D converter, JTAG (IEEE 1149.1 compliant) interface for on-chip debugging, and six software selectable power saving modes. An image of the MCU is shown in Figure \ref{fig:mcu}.

\begin{figure}[tbph]
	\centering
	\includegraphics[width=0.80\linewidth]{"images/mcu"}
	\caption{ATMEL AT90USB 1287 MCU}
	\label{fig:mcu}
\end{figure}

\subsubsection{Conveyor System}
A conveyor belt was used to transport the items that were to be sorted. It was driven by a DC motor and the speed of the conveyor was controlled by the PWM signal. Guides were provided at the loading point of the conveyor which aligned the loaded items on the belt and ensured that there was a sufficient gap between adjacent items for the sensors to read correctly. The belt could be started or stopped as required but the speed of the belt was to be constant during the operation. A model of the system is shown in Figure \ref{fig:model}.

\begin{figure}[tbph]
	\centering
	\includegraphics[width=0.89\linewidth]{"images/model"}
	\caption{Model of the conveyor system and inspection stations}
	\label{fig:model}
\end{figure}

\subsubsection{Inspection Stations}
As the items traveled on the conveyor belt, they passed through various sensor stations which classified the material and visual characteristics of the items. Three sensors were used in this project to determine the type of the items.
\begin{itemize}
	\item Pre-reflectance Optical Sensor:\\
	It is a digital sensor which is active high. It was used to identify if an item was in the range of the reflectance sensor. Only when an object was between the pre-reflectance optical sensor and the reflectance sensor, the analog signal was measured and converted into a digital value.
	\item Reflectance Sensor:\\
	It is an analog sensor which measured the reflectance of the item in its range. Based on the visual and material characteristics, different types of items have their unique range of reflectance value which can be used to identify the type of the item.
\end{itemize}

\subsubsection{Sorting System}
There is an exit optical sensor placed near the end of the conveyor belt system. If an item was detected by the exit optical sensor, which is an active low digital sensor, it indicated that the item has reached the end of travel. There was a rotating bin below the end of travel which could collect the falling items. If the item at the end of travel didn't match the respective tray bin, the belt was stopped and the tray rotated to the correct bin before starting the belt again and allowing the item to fall down. The location of the bin was identified by using a Hall-Effect sensor, which set the bin to its home position before the start of the inspection. The positions of other bins on the tray were defined relative to the home position. The two sensors used in the sorting system can be summarized as below:
\begin{itemize}
	\item Exit Optical Sensor\\
	It is an active low digital sensor which indicates that the item has reached the end of travel.
	\item Hall-Effect Sensor\\
	It is an active low digital sensor that sets the sorting tray to its home position.
\end{itemize}