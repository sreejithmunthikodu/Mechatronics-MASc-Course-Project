\section{Methods and Design Approach}\label{sec:intro}
The design demands a careful coordination of the DC motor, different sensor stations and the stepper motor for correctly sorting the items on the conveyor. The items were loaded onto the conveyor belt which took them through the sensor stations. The first sensor station detected the type of the item and the exit sensor station made sure that the tray was positioned correctly before the item was allowed to fall down into the bin.

The conveyor belt was driven by a DC motor using PWM signal. The speed of the DC motor was controlled by the width of the PWM signal. The speed was optimized for minimum traveling time with the time available for the ADC as a constraint. Having a faster belt speed resulted in overlapping ADC reflectance values for the different objects and hence resulted in identifying incorrect object type. Also, at very high speed, when the belt was stopped at the exit sensor, the object sometimes slipped and fell into the incorrect bin due to the higher inertia caused by the faster belt speed.

The reflectance sensor was primarily used to identify the type of the item. It is an analog sensor and hence Analogue to Digital Conversions were required. A 10-bit ADC resolution was used in this project. Since the ADC consumes considerable process cycles, the pre-reflectance optical sensor was used to ensure that the ADC started only when an item was present in the range of the reflectance sensor. As the items moved through the belt and reached the first sensor station, which was the pre-reflectance optical sensor and the reflectance sensor, ADC was started and the minimum value of the ADC result was stored. As the item left the first sensor station and traveled on its way to the exit sensor, the MCU compared the minimum ADC result value with that of the range of each item to be sorted and identified the object type. A dynamic FIFO linked list was created and the object type was enqueued.

When the object reached the exit optical sensor station, the first item in the linked list was dequeued and the object type was read. The object type is counted using a different variable to update the total number of each item sorted and to identify the number of items remaining on the belt. The object type of the dequeued item was compared with the current tray position. If the tray position matched that of the object type, the item was allowed to fall into the correct bin without stopping the conveyor belt. If the current bin did not match the object type, the belt was stopped and a command was given to the stepper motor to rotate the tray to the correct bin. Once the correct bin was positioned, the belt was started and the object allowed to fall into the correct bin.

The Hall-Effect sensor set the home position of the bin which was black. Other bin positions were defined relative to the home position and the stepper was turned 90 degrees clockwise, 90 degrees counterclockwise or 180 degrees as required to set the bin to the correct position before the object was dropped.

The design is mainly interrupt driven. A large chunk of the code is executed inside the interrupts. The first interrupt is triggered when the item is detected by the pre-reflectance optical sensor. The second interrupt is triggered when an ADC is completed.  A third interrupt is triggered when the object is detected by the exit optical sensor. The fourth and fifth interrupts are used for the system pause button and system ramp-down button respectively.

For the system pause, a push-button triggered an interrupt on rising edge when activated. When the pushbutton is pressed and released, the interrupt stops the conveyor belt immediately and displays the information such as the number of each object type sorted and the total number of items remaining on the belt. When the button is pressed and released again, the belt resumes and the system continues to sort items.

Another pushbutton was used for the system ramp-down function. When the interrupt is triggered by the pushbutton, the systems starts counting and after a set delay the belt is stopped and the sorted item information is displayed on the LCD screen. The delay in stopping the conveyor belt ensures that all items remaining on the belt while the system ramp-down is initiated, are sorted before shutting down the system.
