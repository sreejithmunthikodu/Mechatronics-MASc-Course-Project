\section{Testing and Calibration Procedures}\label{sec:testingcalibration}
Extensive testing and calibration were done on the system prior to the final demonstration to make sure that the system sorted all 48 items with zero errors in the minimum possible time. Calibration of the reflectance proved to be the most critical task in the set up of the system as an error would result in classifying it as the incorrect object type. We used a LCD display for the final demonstration to display the details of items sorted. However, for the calibration of the relectance sensor, we used 8 LEDs on PORTC and the 2 inbuilt LEDs on PORTD as this was only a one-time activity before start-up of the system. A separate code was maintained for calibration. The 10-bit ADC values were stored in a spreadsheet for each item type and after multiple trials, the range of reflectance values corresponding to each object type were identified and updated on the main code.

The reflectance range of steel, aluminum, and the plastics were sufficiently spread apart and hence the system never gave any errors in differentiating between them. However, the range of black plastic and white plastic were very close and hence frequent calibrations were required to make sure that the system identified the item types correctly.

Due to noise created by the DC motor and stepper motor, the ADC and other sensors were sometimes giving false errors and sometimes triggered false interrupts. When discussed with Mr. Patrick Chang, the lab instructor, about these issues, he explained what noise and other interferences can disrupt the system performance and he also explained to us how to remove such noises. We then used low-pass filters to filter out the noises from the sensors which then started giving correct readings without any random errors.

The optimization of the stepper and DC motor speed was also a challenging task. We tried various acceleration values for the stepper. However, at high speeds, the stepper was so fast that the items were falling on the edges. However, later it was found that the tray home position was slightly misaligned. This problem was identified and corrected only during the demonstration and hence we could not use a fast stepper acceleration which compromised on the time required to sort the pieces.

Different conveyor speeds were tried by varying the PWM signal. Finally, an optimum speed of the belt was identified. Beyond this speed, items were slipping off the belt at the exit sensor when the belt was stopped and also, the reflectance sensor was giving less accurate readings due to less available time for ADC to read the reflectance sensor values.
