\section{Novel System Additions}
\subsection*{LCD Display}
The system pause and system ramp-down functions need to display the details of the items sorted in each bin and the number of partially sorted items if any. The project objectives require using LEDs to display these numbers. However, an option is given to display the information on an LCD display which would fetch additional marks. We successfully implemented the LCD display and showed all the required information on the LCD display rather than using the LEDs.

From a technical perspective, the LCD was implemented using the SPI library and connected to PORT C of the MCU, in place of the original 8 LEDs. An open-source library was found from the Internet, which allowed us to easily implement the LCD display. The library code is shown in Listing 12 of Appendix B. 

The LCD was connected to the MCU as shown in the circuit diagram in Figure \ref{fig:circuit}. The LCD and its pins are shown in Figure \ref{fig:lcd} and its pin descriptions are given in Table \ref{table:lcdpins}. 

\begin{figure}[tbph]
	\centering
	\includegraphics[width=0.65\linewidth]{"images/lcd"}
	\caption{LCD pins}
	\label{fig:lcd}
\end{figure}

\begin{table}[h!]
	\centering
	\begin{tabular}{|c|l|} 
		\hline 
		\textbf{Pin} & \multicolumn{1}{|c|}{\textbf{Description}} \\
		\hline
		VSS & Ground (0V) \\ 
		\hline 
		VDD & Supply voltage (5V) \\ 
		\hline 
		VE & Contrast adjustment \\ 
		\hline 
		RS & Register Select; selects the command register or display register \\ 
		\hline 
		RW & Read/Write; selects to write to the register or to read from the register \\ 
		\hline 
		E & Enable; sends data to data pins \\ 
		\hline 
		D0-D7 & 8-bit data pins (only pins D4-D7 were used in the project) \\ 
		\hline 
		K & Backlight cathode; backlight VCC (5V) \\ 
		\hline 
		A & Backlight anode; backlight ground (0V) \\ 
		\hline 
	\end{tabular}
	\caption{LCD pin descriptions}
	\label{table:lcdpins}
\end{table}