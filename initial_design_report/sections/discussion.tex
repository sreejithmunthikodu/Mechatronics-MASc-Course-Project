\section{Discussion}\label{sec:discussion}
The program begins with setting up the home position of the stepper motor using a hall effect sensor. Then stepper motor is stopped and an empty linked list is initiated to store the items that moves through the conveyor. Conveyor belt is now started and items are loaded. When the first input optical sensor detects an object, ferromagnetic sensor is activated to identify if the item is metallic or nonmetallic. Then the item is detected by the pre-reflectance optical sensor and it triggers an ADC at the reflectance sensor. Combined with the readings from reflectance sensor and ferromagnetic sensor, the type of item on the conveyor is now precisely identified and enqueued in the FIFO linked queue. ADC is stopped if the pre-reflectance optical sensor detects no item on the conveyor. 

As the items travel further on the conveyor and the exit optical sensor detects an item at the end of travel, the first item in the linked list is read and the bin position of the stepper motor is compared. If the bin is appropriately positioned to collect the item, the conveyor is continued to run and the item is allowed to fall into the respective bin. The collected item is dequeued from the linked list. If the respective bin is not positioned, conveyor is stopped and the stepper motor is turned to position the bin correctly as per the item at the end of travel. Once the stepper is in its position, conveyor is started and the item is allowed to fall into the bin and collected item is removed from the linked list.

Two switches are provided for system ramp down and system pause. When the system pause switch is activated, it will stop the conveyor and stepper motor and display the number of each items sorted and also the number of items on the conveyor belt. When the system ramp down switch is activated, it will sort all the items on the conveyor and then stop the belt and display the number of each items sorted. 
